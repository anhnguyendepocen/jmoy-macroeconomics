\documentclass[12pt,reqno]{amsart}
\usepackage{mathpazo}
\usepackage[scaled=0.9]{helvet}

\newcommand{\newterm}[1]{\emph{#1}}

\title{Consumption}
\author{Jyotirmoy Bhattacharya}
\address{Ambedkar University Delhi}
\email{jyotirmoy@jyotirmoy.net}
\date{\today}

\begin{document}
\maketitle
\section{Two-period case}
\subsection{Budget constraint}
Consider a consumer who lives for two periods, has an endowment of
$y_1$ and $y_2$ units of goods in the two periods respectively and can
borrow and lend any amount that they like at the real rate of interest
$r$. 

Suppose the consumer consumes $c_1$ in the
first period. Then she will have to take a loan of $c_1-y_1$ to
finance her consumption. (This number can be negative, in which case
the consumer is lending rather than borrowing.) In the next period the
consumer will therefore have to make loan repayments of
$(1+r)(c_1-y_1)$. Assume that the consumer does not want to make any
bequests and cannot die with any outstanding loans, consumption in the
second period must be,
\[c_2=y_2-(1+r)(c_1-y_1)\]
Simplifying and rearranging we have
\begin{equation}\label{eq:two-period-budget}
c_1+\frac{c_2}{1+r}=y_1+\frac{y_2}{1+r}
\end{equation}
This is the budget constraint faced by the consumer. We can interpret
this to mean that the present value of the consumer's consumption
stream must equal the present value of their incomes.

\subsection{Utility maximization}
Suppose the consumer maximises a quasiconcave utility function
$U(c_1,c_2)$ subject to this budget constraint. Then the consumer's
first-order conditions are
\begin{align}
U_1(c_1,c_2)&=\lambda\\
U_2(c_1,c_2)&=\lambda/(1+r)
\end{align}
where $\lambda$ is the Lagrange multiplier corresponding to the budget
constraint and $U_i(c_1,c_2)$ denotes the partial derivative $\partial
U/\partial c_i$. We have explicitly shown the dependence of the
partial derivatives on the value of consumption in both periods. These
first-order conditions along with the budget
constraint~\eqref{eq:two-period-budget} together determines the value
of $c_1$, $c_2$ and~$\lambda$.
\subsection{Comparative statics}
Assuming that consumption in both periods is a normal good, an
increase in either $y_1$ or $y_2$ increases both $c_1$ and $c_2$.

The effects of a change in $r$ are ambiguous. An increase in $r$ makes
consumption in period~$2$ relatively cheap compared to consumption in
period~$1$. Therefore the substitution effect causes $c_1$ to decrease
and $c_2$ to increase. It is traditional to decompose the income
effect into two parts. First, an increase in $r$ reduces the present
value of the consumer's endowments and hence decreases his real
income. Second, an increase in $r$, by making the consumption in
period~$2$ cheaper increases his real income.\footnote{
For more about the Slutsky equation in the case of a consumer with
fixed endowments of goods see section~9.1 in Varian's
\emph{Microeconomic Analysis}, 3rd ed.} The sign of the
resultant of these two effects on consumption depends on whether the
consumer is a net lender in period~1 and a net borrower in period~2 or
vice-versa. In case the consumer is a net lender in period~1 and a net
borrower in period~2 the net income effect is positive. Assuming the
consumption in both periods in a normal good, this means that the
substitution effect and the income effect act in opposite directions
on $c_1$ in this case leading to an ambiguous effect.

\section{Many periods}
Assume that rather than just living for two periods the consumer lives
for $T+1$ periods. Further assume that the real rate of interest takes a
constant value $r$ over the consumer's lifetime. For convenience we
define $\delta=1/(1+r)$. It is also convenient to start time from
period~0 rather than period~1.

\subsection{Budget constraint}
Arguing as before, the consumer's budget constraint is
\begin{equation}\label{eq:many-period-budget}
\sum_{i=0}^T \delta^i c_i = \sum_{i=0}^T \delta^i y_i
\end{equation}

\subsection{Utility function}
We could proceed as before by assuming a utility function
$U(c_0,\ldots,c_T)$ and deriving the first order conditions. However,
because the marginal utility in each period depends on consumption in
all periods it is hard to draw any sharp conclusions at this level of
generality. Therefore we need to impose some restrictions on the form
of the utility functions.

Suppose, for example we assume that the utility function is additively
separable, i.e.

\begin{equation}\label{eq:utility-addsep}
U(c_0,\ldots,c_T)=v_0(c_0)+v_1(c_1)+\cdots+v_T(c_T)
\end{equation}

Then the first-order conditions take the form

\begin{equation}\label{eq:foc-additively-separable}
v_i'(c_i) = \delta^i \lambda \qquad i=0,\ldots,T
\end{equation}
where, as before, $\lambda$ is the Lagrange multiplier corresponding
to the budget constraint.

Sometimes we want to restrict the consumers preferences even further,
by assuming that the different $v_i$ differ from each other by only a
geometric discounting factor.

\begin{equation}\label{eq:utility-geometric}
U(c_0,\ldots,c_T)=\sum_{i=0}^T \beta^i u(c_i)
\end{equation}
where $\beta$ is a constant, referred to as the subjective rate of
discount, such that $0<\beta<1$.  

In this case the first-order conditions take the particularly simple
form
\begin{equation}\label{eq:foc-stationary}
u'(c_i)=\left(\frac{\delta}{\beta}\right)^i \lambda \qquad i=0,\ldots,T
\end{equation}

In case $\delta=\beta$, this implies that $u'(c_i)$ is the same for
all $i$, which, assuming that $u'(\cdot)$ is a strictly decreasing
function, means that $c_i$ is constant for all $i$. The present
period's income does not influence the present period's consumption at
all. Consumption is determined solely by lifetime resources as given
by~\eqref{eq:many-period-budget}.

The case $\delta \neq \beta$ is also instructive. Suppose
$\delta>\beta$. In this case it follows from~\eqref{eq:foc-stationary}
that consumption decreases over time. Formally, this is because if
$\delta>\beta$ then by~\eqref{eq:foc-stationary} $u'(c_i)$ increases
over time, and since $u'(c)$ is a decreasing function of consumption,
this implies that $c$ decreases over time. 

The economic logic behind this result is that $\delta$ is the number of
units of consumption we have to give up at present in order to
purchase one more unit of consumption next period, whereas $\beta$ is
the number of units of marginal utility we are willing to give up at
present in order to have one more unit of marginal utility in the next
period. Suppose we start with the same consumption $c$ in this period and
the next. If we reduce consumption in the next period by a small
amount $\Delta c$ then at the prevailing market prices we can
increase present consumption by $\delta\Delta c$. The increase in
utility from the increase in present consumption is approximately
$u'(c)(\delta\Delta c)$.\footnote{We are using Taylor's theorem:
  $u(c+\delta\Delta c)-u(c) \approx u'(c)(\delta\Delta c)$} The decrease in utility from the reduction in
next period's consumption is approximately $\beta u'(c)(\Delta c)$. The
net change in utility would be $(\delta-\beta)u'(c)(\Delta c)$ which
is positive when $\delta>\beta$. Thus it is beneficial to increase present consumption and
reduce future consumption if we are starting from a position of
equality. Indeed, it will be optimal to increase consumption in the
present period (say period $i$) and decrease consumption in the next
period (period $i+1$) till the following equality between the MRS and
the price ratio is satisfied,
\[\frac{u'(c_{i+1})}{u'(c_i)}=\frac{\delta}{\beta}\]
\subsection{Exogenous variables}
It is possible to unify~\eqref{eq:utility-addsep}
and~\eqref{eq:utility-geometric} by writing
\[v_i(c_i)=\beta^i u(c_i,\xi_i)\]
where $\xi_i$ is an exogenous variable such a the consumer's age or
the number of members in the household. In this case the first-order
conditions become
\[u'(c_i,\xi_i)=\left(\frac{\delta}{\beta}\right)^i \lambda \qquad
i=0,\ldots,T\]
Knowing how $\xi$ affects the marginal utility would
now let us make some predictions regarding the path of consumption.

\subsection{Comparative statics}
Assuming that consumption in every period is a normal good, an
increase in $y_i$ increases every $c_i$.

The effect of an increase in $r$, or equivalently, a decrease in
$\delta$ remains ambiguous because of the same income and substitution
effects as discussed earlier. But for the utility function given
by~\eqref{eq:utility-geometric}, we can say a little
more. From~\eqref{eq:foc-stationary} we can see that a decrease in
$\delta$ means that the \emph{growth rate} of consumption speeds
up. Remember that even in this case we do not have any information
regarding the \emph{level} of consumption in any period since the
level would depend on $\lambda$ which in turn depends on $\delta$.
\section{Dynamic programming}
\subsection{Example: log utility}
Suppose that the consumer maximises
\[\sum_{i=0}^T \log(c_i)\]
subject to
\begin{align*}
  w_0&=\overline{w_0}\\
  w_{t+1}&=R(w_t-c)\qquad \text{for $t=0,\ldots,T$}\\
  w_{T+1}&=0
\end{align*}

What is the value function and policy function for this problem?

Since there can neither be bequests nor outstanding debt at period
$T$, the policy in that period is simple: consume all your
wealth. Denoting the policy function by $g(\cdot)$ and the value
function in period $t$ by $V_t(\cdot)$, we have,
\begin{equation}\label{eq:dplog-T}
g_T(w) = w,\qquad V_T(w_T) = \log(g_T(w))=\log(w)
\end{equation}

Now consider period $T-1$. Bellman's principle of optimality tells us,
\begin{equation}\label{eq:dplog-optim-T1}
\begin{split}
V_{T-1}(w)&=\max_{c} [\log(c) + V_T(R(w-c))]\\
&=\max_{c} [\log(c) +
\log(R(w-c))]\qquad\text{[using~\eqref{eq:dplog-T}]}
\end{split}
\end{equation}

The first-order condition for this maximisation problem is:
\begin{equation*}
\begin{split}
\frac{1}{c}+\frac{-R}{R(w-c)}&=0\\
w-c&=c\\
c=w/2
\end{split}
\end{equation*}

Since the objective function in~\eqref{eq:dplog-optim-T1} is concave
in $c$ (check this!), the first-order condition is sufficient
and gives us our policy function:
\begin{equation*}
g_{T-1}(w)=w/2
\end{equation*}
Substituting this into~\eqref{eq:dplog-optim-T1} we get the value
function,
\begin{equation}\label{eq:dplog-VT1}
\begin{split}
V_{T-1}(w)&=\log(g_{T-1}(w))+V_T(R(w-g_{T-1}(w)))\\
&=\log(w/2)+\log(Rw/2)\\
&=\log(R)+2\log(w/2)
\end{split}
\end{equation}

Now that we know $V_{T-1}$ we could use the Bellman equation relating
$V_{t-2}$ to $V_{t-1}$ to derive $g_{T-2}$ and $V_{t-2}$. If you do
this you will find,
\begin{equation}\label{eq:dplog-VT2}
g_{T-2}(w) = w/3,\qquad V_{T-2}(w)=(1+2)\log(R)+3\log(w/3)
\end{equation}

We could continue like this to find $V_{T-3},\ldots,V_0$. In general
this is precisely what we do. In fact, in most applications of dynamic
programming  it is not possible to express the value function
by a formula in the state variables and the best that we can do is to
use a computer to calculate the value function at a number of possible
values of the state variable using Bellman's equation.

But our present problem is a particularly simple one. Looking
at~\eqref{eq:dplog-VT1} and~\eqref{eq:dplog-VT2} suggests to us the
guess,
\begin{equation}\label{eq:dplog-Vn}
V_{T-n}(w)=\frac{n(n+1)}{2}\log(R)+(n+1)\log\left(\frac{w}{n+1}\right)
\end{equation}
[Remember $1+2+\cdots+n=n(n+1)/2$]

How do we check that our guess is right? We will use the principle of
mathematical induction. By comparing to~\eqref{eq:dplog-VT1} we see
that~\eqref{eq:dplog-Vn} is correct for $n=1$. Suppose that the
equation is true for $n=k$. What then would be $V_{T-(k+1)}$? We once
again write down the Bellman equation

\begin{equation}\label{eq:dplog-optim-induct}
\begin{split}
V_{T-(k+1)}&=\max_c [\log(c)+V_{T-k}(R(w-c))]\\
&=\max_c \left[\log(c)  +
\frac{k(k+1)}{2}\log(R)+(k+1)\log\left(\frac{R(w-c)}{k+1}\right)\right]\\
&
\text{[assuming~\eqref{eq:dplog-Vn}]}
\end{split}
\end{equation}

The first-order condition is:
\begin{equation}\label{eq:dplog-policy}
\begin{split}
  \frac{1}{c} 
  +
  (k+1)\left(\frac{k+1}{R(w-c)}\right)\left(\frac{-R}{k+1}\right)&=0\\
  (k+1)\frac{1}{(w-c)}&=\frac{1}{c}\\
  c&=w/(k+2)
\end{split}
\end{equation}

Substituting this into~\eqref{eq:dplog-optim-induct} we have
\begin{equation}\begin{split}
V_{T-(k+1)}&=\log(c)+\frac{k(k+1)}{2}\log(R)+(k+1)\log\left(\frac{R(w-c)}{k+1}\right)\\
&\text{substituting~\eqref{eq:dplog-policy},}\\
&=\log\left(\frac{w}{k+2}\right)
+\frac{k(k+1)}{2}\log(R)
+(k+1)\log\left(\frac{Rw}{k+2}\right)\\
&=\log\left(\frac{w}{k+2}\right)
+\frac{k(k+1)}{2}\log(R)
+(k+1)\log(R)+(k+1)\log\left(\frac{w}{k+2}\right)\\
&=\frac{(k+2)(k+1)}{2}\log(R)
+(k+2)\log\left(\frac{w}{k+2}\right)\\
\end{split}
\end{equation}
But this is the same as~\eqref{eq:dplog-Vn} for $n=k+1$. We therefore
conclude that if~\eqref{eq:dplog-Vn} is true for $n=k$ it is also true
for $n=k+1$. We have already checked that~\eqref{eq:dplog-Vn} is true
for $n=1$. Hence we conclude by the principle of mathematical
induction that the value function for our dynamic
programming problem is given by~\eqref{eq:dplog-Vn} for
$n=1,\ldots,T$. 

Also, now that we have verified that~\eqref{eq:dplog-Vn} is indeed the
value function of the problem,~\eqref{eq:dplog-policy} gives the
policy function, i.e.
\begin{equation}\label{eq:dplog-policy2}
  g_{T-n}(w)=w/(n+1)
\end{equation}
\section{The Euler equation}
As we discussed in the last section, for most dynamic programming
problems it is not possible to compute the value and policy functions
in terms of simple formulae. The best we can do is to calculate
numerical values. But even if we cannot find an exact formula for the
solution to our optimisation problem, it may still be possible to get
some qualitative information about the problem by studying the
consequences of the Bellman equation. That is the
subject of this section.

Let's recall the Bellman equation,
\[V_t(w_t)=\max_{c_t}[u(c_t)+V_{t+1}(R(w_t-c_t))\]

The first order condition for this maximisation problem is:
\begin{equation}\label{eq:euler-foc}
  u'(c_t) = RV_{t+1}'(R(w_t-c_t))
\end{equation}
By itself~\eqref{eq:euler-foc} does not seem very useful unless we
know $V_{t+1}(\cdot)$ and can calculate its derivative. But there is a
trick that we can use to eliminate this unknown derivative
from~\eqref{eq:euler-foc}.\footnote{The `trick' is a particular case
  of a general result known as the envelope theorem. See section M.L
  of Mas-Colell, Whinston and Green or some mathematical methods book for
  more detail.}

Let $c_t^*(w_t)$ be the optimal consumption in period $t$ when period
$t$ wealth is $w_t$. From~\eqref{eq:euler-foc} we already know that,
\begin{equation}\label{eq:euler-foc-fn}
u'[c_t^*(w_t)]=RV_{t+1}'[R(w_t-c_t^*(w_t))]=RV_{t+1}'(w_{t+1})
\end{equation}

\pagebreak

But from the definition of the value function
\[V_t(w_t) = u[c_t^*(w_t)]+V_{t+1}[R(w_t-c_t^*(w_t))]\]
Differentiating with respect to $w_t$ we have,
\begin{align*}
V_t'(w_t) &=
u'[c_t^*(w)]{c_t^*}'(w_t)
+V_{t+1}'[R(w_t-c_t^*(w_t))][R(1-{c_t^*}'(w_t))]\\
&={c_t^*}'(w_t)[u'(\cdot))-RV_{t+1}'(\cdot)]+RV_{t+1}'[R(w_t-c_t^*(w_t))]\\
\intertext{From \eqref{eq:euler-foc-fn} the first terms equals $0$,
  so,}
V_t'(w_t)&=RV_{t+1}'(w_{t+1})\\
\intertext{Using~\eqref{eq:euler-foc-fn}}
V_t'(w_t)&=u'(c_t)
\end{align*}

The equation above was derived for arbitrary $t$. So it is equally
good for $t+1$, i.e.
\[V_{t+1}'(w_{t+1})=u'(c_{t+1})\]
Substituting this in~\eqref{eq:euler-foc-fn} we have,
\begin{equation}\label{eq:euler}
\frac{u'(c_{t+1})}{u'(c_t)}=\frac{1}{R}=\delta
\end{equation}

This condition is known as the Euler\footnote{Pronounced
  ``oiler''. Named after a eighteenth-century mathematician who was
  among the earliest to study dynamic optimisation problems.} equation for our dynamic
programming problem. We can alternatively derive it by starting out
with an optimal consumption plan, increasing consumption in period $t$
by a small amount $\Delta c$ and reducing consumption in period
$t+1$ by $R\Delta c$ so that wealth at the end of the period
$t+1$ is once again the same as what it would have been under the
optimal plan. The first-order change in utility from this deviation is
\[\Delta u=u'(c_t)[\Delta c]-u'(c_{t+1})[R\Delta c]\]
Now for the original plan to have been optimal $\Delta u$ must be $0$
since if $\Delta u>0$ the deviation considered above increases total
utility whereas if $\Delta u<0$ then the opposite of the deviation
considered above increases total utility. But $\Delta u=0$ implies 
\[u'(c_t)-Ru'(c_{t+1})=0\]
which is again our Euler equation~\eqref{eq:euler}.

The Euler equation also follows from the first-order
conditions~~\eqref{eq:foc-additively-separable} of the Lagrange-multiplier
approach, showing that we have
come full circle.

The Euler equation tells us how consumption should grow or decline. It
does not tell us the level of the consumption. But we can characterise
the entire consumption path if we keep track of
the path of wealth implied by the path of consumption and impose, in
addition to the Euler equation, the
conditions
\[w_0 = \overline{w_0}\]
which comes to us as a given data and 
\[w_T=0\]
which comes to us from our no bequest, no terminal borrowing,
monotonic utility assumptions about the terminal period.
\end{document}